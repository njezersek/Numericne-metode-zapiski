\documentclass[a4paper,8pt]{extarticle}
\usepackage[utf8]{inputenc}

\usepackage{fancyhdr}

\usepackage[pdftex]{graphicx} % Required for including pictures
\usepackage[pdftex,linkcolor=black,pdfborder={0 0 0}]{hyperref} % Format links for pdf
\usepackage{calc} % To reset the counter in the document after title page
\usepackage{enumitem} % Includes lists

\usepackage{textcomp}
\usepackage{eurosym}

\usepackage{ dsfont } % font za množice
% tabele
\usepackage{array}
\usepackage{wrapfig}

\usepackage{tikz,forest}
\usetikzlibrary{arrows.meta}

\frenchspacing % No double spacing between sentences
\setlength{\parindent}{0pt}
\setlength{\parskip}{0.7em}

\usepackage{mathtools}
\usepackage{blkarray, bigstrut} %


\usepackage{amssymb,amsmath,amsthm,amsfonts}
\usepackage{multicol,multirow}
\usepackage{calc}
\usepackage{ifthen}
\usepackage{tabularx}
\usepackage[landscape]{geometry}
\usepackage{listings}
\usepackage{inconsolata}
%\usepackage[colorlinks=true,citecolor=blue,linkcolor=blue]{hyperref}
%\usepackage{accents}

\newcommand{\vect}[1]{\accentset{\rightharpoonup}{#1}}

\ifthenelse{\lengthtest { \paperwidth = 11in}}
    { \geometry{top=.5in,left=.5in,right=.5in,bottom=.5in} }
	{\ifthenelse{ \lengthtest{ \paperwidth = 297mm}}
		{\geometry{top=1cm,left=1cm,right=1cm,bottom=1cm} }
		{\geometry{top=1cm,left=1cm,right=1cm,bottom=1cm} }
	}
\pagestyle{empty}
\makeatletter
\renewcommand{\section}{\@startsection{section}{1}{0mm}%
                                {-1ex plus -.5ex minus -.2ex}%
                                {0.5ex plus .2ex}%x
                                {\normalfont\large\bfseries}}
\renewcommand{\subsection}{\@startsection{subsection}{2}{0mm}%
                                {-1explus -.5ex minus -.2ex}%
                                {0.5ex plus .2ex}%
                                {\normalfont\normalsize\bfseries}}
\renewcommand{\subsubsection}{\@startsection{subsubsection}{3}{0mm}%
                                {-1ex plus -.5ex minus -.2ex}%
                                {1ex plus .2ex}%
                                {\normalfont\small\bfseries}}
\makeatother
\setcounter{secnumdepth}{0}
%\setlength{\parindent}{0pt}
%\setlength{\parskip}{0pt plus 0.5ex}

% listings okolje za psevdo kodo
\lstnewenvironment{koda}[1][] %defines the algorithm listing environment
{   
    \lstset{ %this is the stype
        mathescape=true,
        %basicstyle=\scriptsize, 
		columns=flexible,
        keywordstyle=\bfseries\em,
        keywords={,vhod, izhod, zacetek, konec, koncamo, ponavljaj, dokler, ce, vrni, za, vsak, vse, v, sicer,} %add the keywords you want, or load a language as Rubens explains in his comment above.
        xleftmargin=.1\textwidth,
		tabsize=4,
		%frame=leftline,xleftmargin=5pt,xrightmargin=5pt,framesep=5pt,
		%inputencoding = utf8,
		extendedchars = true,
		literate={ž}{{\ˇz}}1 {š}{{\ˇs}}1 {č}{{\ˇc}}1 {Ž}{{\ˇZ}}1 {Š}{{\ˇS}}1 {Č}{{\ˇC}}1,
        #1 % this is to add specific settings to an usage of this environment (for instnce, the caption and referable label)
    }
}
{}
% -----------------------------------------------------------------------


\newcolumntype{C}[1]{>{\centering\arraybackslash$}m{#1}<{$}}
\newlength{\mycolwd}                                         % array column width
\settowidth{\mycolwd}{$e^{-\frac{i}{\hbar}|A|t}$}% "width" of $e^{-\frac{i}{\hbar}|A|t$; largest element in array
\begin{document} 

\begin{multicols}{4}
\setlength{\premulticols}{1pt}
\setlength{\postmulticols}{1pt}
\setlength{\multicolsep}{1pt}
\setlength{\columnsep}{2pt}

\section{Premična pika}
\[ x = (-1)^S m \cdot b^e\]
\[ m = c_0 + c_1 b^{-1} + c_2 b^{-2} + \dots + c_t b^{-t} \]
\begin{align*}
	S\quad \dots \quad & \text{predznak} \\
	b\quad \dots \quad & \text{baza, ponavadi 2} \\
	m\quad \dots \quad & \text{vrednost mantise} \\
	t\quad \dots \quad & \text{dolžina mantise} \\
	e\quad \dots \quad & \text{vrednost eksponenta $L \leq e \leq U$} \\
	c_i\quad \dots \quad & \text{števke v mejah $0 \leq c_i \leq b-1$} \\
\end{align*}

Sistem premične pike označimo z $P(b, t, L, U)$.

\subsection{Standard IEEE}
Eksponent je zapisan z odmikom:
\[ E = e + \text{odmik}\]
Če je $E=0$, uporabimo \textbf{denormiran zapis}:
\[ x = (-1)^S (c_1 b^{-1} + c_2 b^{-2} + \dots + c_t b^{-t}) \cdot b^{e+1}\]
Sicer pa \textbf{normiran zapis}:
\[ x = (-1)^S (1 + c_1 b^{-1} + c_2 b^{-2} + \dots + c_t b^{-t}) \cdot b^{e}\]

Če so vsi biti eksponenta 1 in vsi biti mantise 0, je $x = (-1)^S \infty$.

Če so vis biti eksponenta 1 in vsi biti mantise niso 0, je $x = \text{NaN}$.

\begin{itemize}
	\item \textbf{Single precision} $b=2$, $t=23$, $L=-126$, $U=127$,
	odmik: 127
	\begin{center}
		\begin{tabular}{ |c|c|c| } 
			\hline
			\textit{predznak} 1 & \textit{exponent} 8 & \textit{mantisa} 23 \\ 
			\hline
		\end{tabular}
	\end{center}
	\item \textbf{Double precision} $b=2$, $t=52$, $L=-1022$, $U=1023$, 
	odmik: 1023
	\begin{center}
		\begin{tabular}{ |c|c|c| } 
		 \hline
		 \textit{predznak} 1 & \textit{exponent} 11 & \textit{mantisa} 52 \\ 
		 \hline
		\end{tabular}
	\end{center}
\end{itemize}

\subsection{Zaokroževanje}
Naj bo $x$ pozitivno število z neskončnim zapisom
\[ x = (c_1 b^{-1} + c_2 b^{-2} + \dots + c_t b^{-t} + c_{t+1} b^{-t-1})  + \dots b^e \]

Kandidata za približek $\text{fl}(x)$ sta:
\begin{align*}
	x_- =&\ (c_1 b^{-1} + c_2 b^{-2} + \dots + c_t b^{-t}) b^e \\
	x_+ =&\ (c_1 b^{-1} + c_2 b^{-2} + \dots + c_t b^{-t} + b^{-t}) b^e
\end{align*}

Vzamemo tistega, ki je bljižje. Če sta enako blizu, izberemo tistega, ki ima zadnjo števko sodo.

\subsubsection{Osnovna zaokrožitvena napaka}
\[ u = \frac{1}{2} b^{-t} \]
\[ \text{fl}(x) = x(1+\delta) \quad \text{za} \quad |\delta| \leq u \]
\[ \frac{|\text{fl}(x) - x|}{|x|} \leq u\]

\section{Napake pri numeričnem računanju}
\begin{itemize}
	\item \textbf{Neodstranljiva napaka} Namesto $x$ imamo približek $\bar{x}$.
	\[ D_n = f(x) - f(\bar{x}) \]
	\item \textbf{Napaka metode} Namesto funkcije $f$ imamo približek $g$.
	\[ D_m = f(\bar{x}) - g(\bar{x})\]
	\item \textbf{Zaokrožitvena napaka} Pri računanju $\tilde{y} = f(\bar{x})$ se pri vsaki operaciji se pojavi zaokrožitvena napaka. Namesto $\tilde{y}$ dobimo $\hat{y}$.
	\[ D_z = \tilde{y} - \hat{y}\]
\end{itemize}

Celotna napaka je $D = |D_n| + |D_m| + |D_z|$.

\subsection{Stopnja občutljivosti}
\textit{Razmerje velikosti spremembe podatkov in spremembe rezultata.}

Naj bo $f: \mathbb{R} \to \mathbb{R}$ zvezno odvedlijva funkcija in $\delta x$ majhna motnja.
\begin{itemize}
	\item \textbf{Absolutna občutljivost} $f$ v točki $x$:
	\[ | f(x + \delta x) - f(x) | \approx |f'(x)| \cdot |\delta x| \]
	\item \textbf{Relativna občutljivost} $f$ v točki $x$:
	\[ \frac{| f(x + \delta x) - f(x) |}{|f(x)|} \approx \frac{|f'(x)| \cdot |\delta x|}{|f(x)|} \]
\end{itemize}

\subsection{Obratna in direktna stabilnost}
\begin{itemize}
	\item \textbf{Direktna stabilnost}: za vsak $x$ direktna napaka ($|f(x) - f(x + \Delta x)|$) majhna (absolutno oz. relativno).
	\item \textbf{Obratna stabilnost}: za vsak $x$ razlika $\Delta x$, ki bi nam dala pravi rezultat majhna.
\end{itemize}

\[ |\text{direktna napaka}| \lesssim \text{občutlijvost} \cdot |\text{obratna napaka}|\]

\section{Nelinearne enačbe}
Iščemo ničle funkcije $f$.
\begin{itemize}
	\item \textbf{Enostavne ničle}: 
	\[f(\alpha) = 0 \quad \text{in} \quad f'(\alpha) \neq 0\]
	\item \textbf{m-kratne ničle}: 
	\[f(\alpha) = f'(\alpha) = \dots = f^{(m-1)}(\alpha) = 0\]
\end{itemize}

\subsection{Občutljivost ničle}
Naj bo $\alpha$  $m$-kratna ničla $\hat{\alpha}$ približek, da je $f(\hat{\alpha}) = \varepsilon$.

Če $f$ razvijemo v Taylorjevo vrsto okoli $\alpha$ in vzamemo prvih $m+1$ členov dobimo:

\begin{align*}
	\varepsilon &\doteq \frac{f^{(m)} (\alpha)}{ m! }(\hat{\alpha} - \alpha )^m &
	|\hat{\alpha} - \alpha| &\doteq \sqrt[m]{\frac{\varepsilon \cdot m!}{|f^{(m)} (\alpha)|}}
\end{align*}


\subsection{Bisekcija}

\begin{koda}[]
vhod: funkcija $f: [a,b] \to \mathbb{R}$, $f(a)f(b) < 0$, natancnost $\varepsilon$
izhod: nicla funkcije $f$

dokler $|b-a| > \varepsilon$:
	$c \leftarrow \frac{a+b}{2}$
	ce $\text{sign}(f(c)) = \text{sign}(f(a))$:
		$a \leftarrow c$
	sicer:
		$b \leftarrow c$
vrni $c$
\end{koda}

$c$ je približek ničle $\alpha$. Velja
\[ |\alpha - c| \leq \frac{b-a}{2^m} \leq \varepsilon\]

Za natančnost $\varepsilon$ potrebujemo  $\log_2 \left( \frac{b-a}{\varepsilon}\right)$ korakov.


\subsection{Navadna iteracija}
Rešujemo $f(x) = 0$. Enačbo pretvorimo v $x = g(x)$. Načinov je veliko:
\begin{itemize}
	\item $g(x) = f(x) + x$
	\item $g(x) = cf(x) + x$
	\item $g(x) = h(x)f(x) + x$ kjer je $h(x)$ funkcija, ki nima ničle v $\alpha$.
\end{itemize}

\subsubsection{Izrek o konvergenci navadne iteracije}
Naj bo $\alpha$ negibna točka za $g$ in naj $g$ na intervalu $[\alpha - d, \alpha + d]$ ($d > 0$)
zadošča Lipschitzovemu pogoju:
\[ \exists m \in [0, 1)\ \forall x,y \in I\ : |g(x) - g(y)| \leq m |x-y| \]
tedaj je $g$ \textit{skrčitev} na $I$.

Potem za vsak $x_0 \in I$ zaporedje $x_{r+1} = g(x_r)$ konvergira k $\alpha$ in velja:
\[ |x_r - \alpha| \leq \frac{m}{1-m} |x_r - x_{r-1}| \]

\textit{Posledica}: Če je $g$ zvezno odvedljiva v $\alpha$ in velja $g'(\alpha) < 1$, obstaja interval $I$, ki vsebuje $\alpha$, da za vsak $x_0 \in I$ zaporedje konvergira k $\alpha$.

\begin{itemize}
	\item Če je $|g'(\alpha)| < 1$, je $\alpha$ \textit{privlačna negibna točka}
	\item Če je $|g'(\alpha)| > 1$, je $\alpha$ \textit{odbojna negibna točka}
\end{itemize}

Če je $\alpha$ odbojna za $g$, je privlačna za $g^{-1}$:\\ $g(x) = x \implies x = g^{-1}(x)$

\subsubsection{Hitrost konvergence}
$p > 0$ je red konvergence, če $\exists C_1, C_2 > 0$, da za vse dovolj pozne člene zaporedja $x_{r+1} = g(x_r)$ velja:
\[ C_1 |x_r - \alpha|^p \leq |x_{r+1} - \alpha| \leq C_2 |x_r - \alpha|^p \]
\textit{Vsak korak se št. decimalk pomnoži s $p$.}

Naj bo $g$ v okolici $\alpha$ $p$-krat zvezno odvedlijva in naj velja 
$g(\alpha) = \alpha$, $g'(\alpha) = \dots = g^{(p-1)}(\alpha) = 0$ in $g^{(p)}(\alpha) \neq 0$.
Tedaj je red konvergence enak $p$.

\subsubsection{Standardni redi konvergence}
\begin{align*}
	p=1 & \quad \dots \quad \text{linearna konvergenca} \\
	p=2 & \quad \dots \quad \text{kvadratična konvergenca}
\end{align*}


\subsection{Tangentna (Newtonova) metoda}
\[ x_{r+1} = x_r - \frac{f(x_r)}{x_r}\]
Red konvergence:
\begin{itemize}
	\item 2, če je $\alpha$ enkratna ničla $f'(\alpha) \neq 0$
	\item 3, če je $f'(\alpha) \neq 0$ in $f''(\alpha) = 0$
	\item 1, če je $\alpha$ večkratna ničla $f'(\alpha) = 0$
\end{itemize}

\subsection{Sekantna metoda}
\[ x_{r+1} = x_r - \frac{f(x_r)}{ \frac{f(x_r)-f(x_{r-1})}{x_r - x_{r-1}} }\]

Red konvergence: $\frac{\sqrt{5}+1}{2} \approx 1.62$

\subsection{Metoda $(f,f',f'')$}
\[ x_{r+1} = x_r - \frac{f(x_r)}{f'(x_r)} - \frac{f''(x_r) f^2(x_r)}{2f'^3(x_r)}\]

Red konvergence: 3 \textit{(pri predpostavkah)}

\subsection{Müllerjeva metoda}
Na $x_r$, $x_{r-1}$, $x_{r-2}$ napnemo parabolo, ničla parabole je naslednji približek.

\[ p(x) = a(x - x_r)^2 + b(x-x_r) + c\]
\[ x_{r+1} = x_r - \frac{2c}{b+\text{sign}(b) \sqrt{b^2-4ac}}\]

\subsection{Hallejeva metoda}
\[ x_{r+1} = x_r - \frac{2 f(x_r) f'(x_r)}{2 f'(x_r)^2 - f(x) f''(x)}\]


\end{multicols}
\end{document}